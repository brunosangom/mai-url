\documentclass[a4paper,10pt,twocolumn]{article}

\usepackage[utf8]{inputenc}
\usepackage{amsmath}
\usepackage{graphicx}
\usepackage{hyperref}
\usepackage{float}
\usepackage[margin=0.5in]{geometry}
\usepackage{subcaption}

\title{URL Coursework 1: Clustering \\ \Large{Agglomerative clustering via maximum \\ incremental path integral~\cite{PIC}}}
\author{Bruno Sánchez Gómez}
\date{\today}

\begin{document}

\twocolumn[
\maketitle
\begin{abstract}
    This report presents a comprehensive reproduction of the Path Integral Clustering (PIC) algorithm introduced by Zhang et al \cite{PIC}. PIC leverages concepts from statistical physics to effectively cluster data with complex manifold structures by measuring connectivity through path integrals. We implement the algorithm and evaluate it against eleven state-of-the-art clustering methods on both synthetic and real-world imagery datasets (MNIST, USPS, and Caltech-256). Our experiments confirm PIC's exceptional performance, achieving the highest Normalized Mutual Information scores on MNIST ($0.940$) and Caltech-256 ($0.653$), while showing competitive results on USPS. We extend the evaluation to include additional datasets (Iris and Breast Cancer) and metrics (Silhouette score), providing insights into PIC's strengths with non-convex clusters. Scalability analysis reveals that PIC's runtime increases primarily with sample count rather than dimensionality. Despite some reproduction challenges due to implementation ambiguities, our results validate PIC as a significant advancement in clustering methodology, particularly for datasets with complex structures.
\end{abstract}
\vspace{2em}
]

\section{Introduction}
Clustering is a fundamental task in unsupervised learning that aims to partition data into groups where objects within the same cluster are more similar to each other than to those in other clusters. Despite numerous algorithms available, clustering remains challenging, particularly for data with complex manifold structures or non-convex clusters that traditional methods struggle to identify correctly.

Path Integral Clustering (PIC) \cite{PIC} addresses these limitations by leveraging concepts from statistical physics to model the connectivity between data points through path integrals. This innovative approach enables effective clustering of data with arbitrary shapes by measuring how strongly connected pairs of points are through all possible paths between them, rather than relying solely on direct distances.

This report reproduces and extends the experimental evaluation of PIC as introduced in the original paper by Zhang et al \cite{PIC}. We implement the algorithm and compare its performance against eleven state-of-the-art clustering methods on both synthetic datasets and real-world imagery datasets, including MNIST, USPS, and Caltech-256. Additionally, we extend the evaluation to new datasets and metrics to further assess PIC's capabilities and limitations.

The remainder of this report is organized as follows: Section 2 presents the theoretical foundations and implementation details of the PIC algorithm. Section 3 describes our experimental setup, results, challenges encountered during reproduction, and additional experiments. Section 4 concludes with a summary of findings and potential directions for future work.

\section{Path Integral Clustering}

\subsection{PIC Algorithm}
Explain the algorithm itself.

\subsection{Implementation}
Explain the algorithm implementation here.


\section{Experiments}
This section presents the experimental evaluation of the Path Integral Clustering (PIC) algorithm compared to various state-of-the-art clustering methods. We assess performance on both synthetic and real-world imagery datasets, following the experimental framework described in the original paper\cite{PIC}. Then, we present some criticism and issues encountered during the experiment replication process. Finally, we perform some additional experiments to further evaluate PIC's performance.

\subsection{Synthetic Datasets}
We recreated the three synthetic datasets introduced in \cite{PIC} to visually demonstrate PIC's effectiveness on data with complex structures.

Figure \ref{fig:synthetic} shows the clustering results on these synthetic datasets.

\begin{figure*}[htb]
    \centering
    \begin{subfigure}[b]{0.32\textwidth}
        \centering
        \includegraphics[width=\textwidth]{../results/plots/dataset_0/A-link_clustering.png}
    \end{subfigure}
    \hfill
    \begin{subfigure}[b]{0.32\textwidth}
        \centering
        \includegraphics[width=\textwidth]{../results/plots/dataset_0/S-link_clustering.png}
    \end{subfigure}
    \hfill
    \begin{subfigure}[b]{0.32\textwidth}
        \centering
        \includegraphics[width=\textwidth]{../results/plots/dataset_0/PIC_clustering.png}
    \end{subfigure}

    \begin{subfigure}[b]{0.32\textwidth}
        \centering
        \includegraphics[width=\textwidth]{../results/plots/dataset_1/A-link_clustering.png}
    \end{subfigure}
    \hfill
    \begin{subfigure}[b]{0.32\textwidth}
        \centering
        \includegraphics[width=\textwidth]{../results/plots/dataset_1/S-link_clustering.png}
    \end{subfigure}
    \hfill
    \begin{subfigure}[b]{0.32\textwidth}
        \centering
        \includegraphics[width=\textwidth]{../results/plots/dataset_1/PIC_clustering.png}
    \end{subfigure}

    \begin{subfigure}[b]{0.32\textwidth}
        \centering
        \includegraphics[width=\textwidth]{../results/plots/dataset_2/NJW_clustering.png}
    \end{subfigure}
    \hfill
    \begin{subfigure}[b]{0.32\textwidth}
        \centering
        \includegraphics[width=\textwidth]{../results/plots/dataset_2/NCuts_clustering.png}
    \end{subfigure}
    \hfill
    \begin{subfigure}[b]{0.32\textwidth}
        \centering
        \includegraphics[width=\textwidth]{../results/plots/dataset_2/PIC_clustering.png}
    \end{subfigure}

    \caption{Clustering results on the three Synthetic Datasets (NMI scores in parentheses).}
    \label{fig:synthetic}
\end{figure*}

The results demonstrate PIC's ability to handle complex data structures. Particularly noteworthy is PIC's performance on Dataset 1, where it successfully identified both the dense clusters and the circular pattern despite the presence of noise. For Dataset 2, PIC effectively captured the sinusoidal and non-convex patterns, outperforming traditional algorithms that typically prefer convex clusters. Dataset 3 showcases PIC's robustness to noise, accurately separating the two main clusters from the surrounding noise.

\subsection{Imagery Datasets}
We evaluated PIC and 11 other clustering algorithms on three widely used image datasets:

\begin{enumerate}
    \item \textbf{MNIST}: Handwritten digits (0-4), with 5,139 samples and 784 dimensions (28$\times$28 pixels).
    \item \textbf{USPS}: Handwritten digits (0-9), with 9,298 samples and 256 dimensions (16$\times$16 pixels).
    \item \textbf{Caltech-256}: Reduced to six classes (hibiscus, ketch-101, leopards-101, motorbikes-101, airplanes-101, and faces-easy-101), with 600 samples and 4,200 dimensions (60$\times$70 grayscale images).
\end{enumerate}

\begin{table}[h]
\centering
\begin{tabular}{|l|c|c|c|}
\hline
\textit{NMI} & \textbf{MNIST} & \textbf{USPS} & \textbf{Caltech-256} \\
\hline
\textbf{PIC}       & \textbf{0.940} & 0.835 & \textbf{0.653} \\ \hline
\textbf{k-med}     & 0.318 & 0.553 & 0.315 \\ \hline
\textbf{A-link}    & 0.408 & 0.139 & 0.313 \\ \hline
\textbf{S-link}    & 0.002 & 0.002 & 0.019 \\ \hline
\textbf{C-link}    & 0.539 & 0.374 & 0.395 \\ \hline
\textbf{AP}        & 0.426 & 0.525 & 0.492 \\ \hline
\textbf{NCuts}     & 0.807 & 0.772 & 0.589 \\ \hline
\textbf{NJW}       & 0.898 & 0.784 & 0.529 \\ \hline
\textbf{CT}        & 0.634 & 0.439 & 0.181 \\ \hline
\textbf{Zell}      & 0.913 & \textbf{0.846} & 0.343 \\ \hline
\textbf{C-kernel}  & 0.780 & 0.768 & 0.521 \\ \hline
\textbf{D-kernel}  & 0.903 & \textbf{0.846} & 0.508 \\
\hline
\end{tabular}
\caption{Normalized Mutual Information (NMI) scores for all algorithms on image datasets. Higher values indicate better performance. Bold indicates best performance.}
\label{table:nmi}
\end{table}

\begin{table}[h]
\centering
\begin{tabular}{|l|c|c|c|}
\hline
\textit{CE} & \textbf{MNIST} & \textbf{USPS} & \textbf{Caltech-256} \\
\hline
\textbf{PIC}       & \textbf{0.016} & 0.269 & 0.307 \\ \hline
\textbf{k-med}     & 0.534 & 0.373 & 0.607 \\ \hline
\textbf{A-link}    & 0.573 & 0.778 & 0.665 \\ \hline
\textbf{S-link}    & 0.779 & 0.833 & 0.828 \\ \hline
\textbf{C-link}    & 0.280 & 0.601 & 0.507 \\ \hline
\textbf{AP}        & 0.960 & 0.934 & 0.705 \\ \hline
\textbf{NCuts}     & 0.115 & 0.356 & 0.328 \\ \hline
\textbf{NJW}       & 0.033 & 0.269 & \textbf{0.290} \\ \hline
\textbf{CT}        & 0.493 & 0.615 & 0.747 \\ \hline
\textbf{Zell}      & 0.027 & 0.197 & 0.680 \\ \hline
\textbf{C-kernel}  & 0.129 & 0.269 & 0.368 \\ \hline
\textbf{D-kernel}  & 0.029 & \textbf{0.132} & 0.315 \\
\hline
\end{tabular}
\caption{Clustering Error (CE) scores for all algorithms on image datasets. Lower values indicate better performance. Bold indicates best performance.}
\label{table:ce}
\end{table}

Looking at the results in Tables \ref{table:nmi} and \ref{table:ce}, we observe that:

\begin{itemize}
    \item \textbf{MNIST}: PIC achieved the highest NMI (0.940) and lowest CE (0.016), significantly outperforming other methods. This suggests that PIC effectively captures the intrinsic manifold structure of the handwritten digits.
    
    \item \textbf{USPS}: PIC performed well with an NMI of 0.835, though slightly behind the Diffusion kernel (D-kernel) and Zell methods which both achieved an NMI of 0.846. In terms of CE, D-kernel had the best performance (0.132), followed by Zell (0.197) and PIC (0.269). These results still show strong performance for PIC.
    
    \item \textbf{Caltech-256}: PIC significantly outperformed all other methods with an NMI of 0.653. For CE, NJW had the lowest value (0.290), followed closely by PIC (0.307). This demonstrates PIC's ability to handle higher-dimensional image data with complex visual patterns.
\end{itemize}

Our results largely align with those reported in the original paper, with PIC consistently performing as one of the top methods across datasets. However, we observed some differences:

\begin{enumerate}
    \item On the USPS dataset, our implementation shows D-kernel and Zell slightly outperforming PIC, whereas the original paper reported PIC as the best method. This discrepancy might be due to differences in the dataset composition (9,298 samples in our case versus 11,000 mentioned in the paper), or in each of our custom implementations (since these algorithms are not supported by stardard libraries).
    
    \item For Caltech-256, we achieved similar relative performance between methods, though our absolute scores differ from the paper, likely due to different preprocessing approaches.
\end{enumerate}

\subsection{Issues Encountered}
Several challenges were encountered during the attempt to replicate the original paper's experiments with the highest fidelity possible:

\begin{enumerate}
    \item \textbf{Synthetic Dataset Generation}: The original paper did not provide clear guidelines for synthetic dataset generation. Considerable tuning was required to create datasets in which the PIC algorithm exhibited the behaviors described in the paper.
    
    \item \textbf{Algorithm Implementation}: Implementing all 11 comparison algorithms was challenging, requiring adaptation of existing libraries and development of custom implementations, since not all of them are supported by stardard Python libraries.
    
    \item \textbf{Dataset Availability}: Two datasets mentioned in the original paper were not available: FRGC-T requires restricted access, and PubFig is no longer publicly available.
    
    \item \textbf{Dataset Discrepancies}: The USPS dataset contained 9,298 samples instead of the 11,000 mentioned in the paper.
    
    \item \textbf{Preprocessing Ambiguity}: For Caltech-256, the paper stated a dimensionality of 4,200 but did not specify how images of different sizes were processed. We adopted a 60$\times$70 grayscale representation.
\end{enumerate}

Despite these challenges, our implementation successfully reproduced the main findings of the original paper, confirming PIC's effectiveness for clustering tasks, especially on datasets with complex manifold structures.

\subsection{Additional Experiments}
To further evaluate PIC's performance, we conducted some additional experiments. First, we gathered time-related information from the imagery datasets experiments in order to perform a scalability analysis.

\subsubsection{Scalability Analysis}
The original paper states that the PIC algorithm scales linearly with the number of clusters. However, it does not mention the algorithm's scalability with respect to the number of samples or dimensions. To investifate this, we recorded the runtime of PIC on each of the three image datasets, which have different number of samples, clusters and dimensions. The results are shown in Table \ref{table:time}.

\begin{table}[h]
    \centering
    \begin{tabular}{|l|c|c|c|}
    \hline
     & \textbf{MNIST} & \textbf{USPS} & \textbf{Caltech-256} \\
    \hline
    \textbf{\# of samples} & 5,139 & 9,298 & 600 \\ \hline
    \textbf{\# of clusters} & 5 & 10 & 6 \\ \hline
    \textbf{Dimensionality} & 784 & 256 & 4,200 \\ \hline
    \textbf{PIC Runtime (s)} & 256.1 & 854.7 & 1.5 \\ \hline
    \end{tabular}
    \caption{Runtime of PIC on different datasets with varying sizes and dimensionalities.}
    \label{table:time}
\end{table}

The results show that PIC's runtime does not seem to be affected by the dimensionality of the dataset in any significant way, since the Caltech-256 dataset has the highest number of dimensions and yet has the lowest runtime by a wide margin. However, the runtime does greatly increase with the number of samples, as seen in the MNIST and USPS datasets. The number of clusters could also have an impact, but it does not seem to be as significant as that of the number of samples.



\section{Conclusion}

This report has presented a thorough reproduction and extension of the Path Integral Clustering (PIC) algorithm as introduced by Zhang et al \cite{PIC}. Our experiments confirm PIC's effectiveness across diverse datasets, particularly highlighting its strengths in handling complex manifold structures and non-convex clusters. Our key findings can be summarized as follows:

\begin{enumerate}
    \item \textbf{Exceptional performance on synthetic data}: PIC demonstrated superior capabilities in identifying clusters with complex geometries on synthetic datasets, successfully handling a wide variety of cluster shapes despite noise.

    \item \textbf{Strong results on image datasets}: PIC achieved the best performance on MNIST (NMI: 0.940) and Caltech-256 (NMI: 0.653), while showing competitive results on USPS.

    \item \textbf{Versatility across dataset types}: Additional experiments on Iris and Breast Cancer datasets showed PIC's adaptability to lower-dimensional, well-structured data, achieving the best scores on Iris and competitive results on Breast Cancer.

    \item \textbf{Evaluation metric insights}: While PIC excelled in external metrics (NMI, CE), it didn't consistently produce the highest Silhouette scores. This highlights PIC's focus on capturing true data distribution through non-convex clusters rather than optimizing for convexity-based measures.

    \item \textbf{Scalability characteristics}: PIC's runtime increases primarily with the number of samples rather than dimensions, with USPS (9,298 samples) requiring 645 seconds compared to only 1.1 seconds for Caltech-256 (600 samples) despite the latter having significantly higher dimensionality.
\end{enumerate}

Despite our general success in reproducing the implementation and experiments of the original paper, we encountered some discrepancies with the reported results, which we attributed to differences in some implementation details that were not explicitly discussed and therefore cannot be reproduced with exact precision.

In addition, during the development of this study we identified some interesting directions for future research:

\begin{enumerate}
    \item \textbf{Application extensions}: Evaluating PIC on even more diverse data types including time series, text, and network data.

    \item \textbf{Hybrid approaches}: Combining PIC with other clustering methods to leverage complementary strengths.

    \item \textbf{Adaptive parameters}: Investigating adaptive methods for systematically setting PIC's hyperparameters to improve its performance on different datasets.
\end{enumerate}

In conclusion, PIC has proven that it could potentially represent a significant advancement in clustering methodology, particularly for datasets with complex structure. We believe that our extended experiments further validate PIC's versatility, while highlighting specific areas where the algorithm could be enhanced to broaden its applicability.

\bibliographystyle{unsrt}
\bibliography{references}

\end{document}