\documentclass[pdf]{beamer}
\usepackage[utf8]{inputenc}
\usepackage{cite}
\usepackage{graphicx}
\usepackage{float}
\usepackage{fontawesome}
\usepackage{multicol}
\usepackage{listings}
\usepackage{array}
\usepackage{subcaption}
\usepackage{multirow}
\usepackage{amsmath}
\usepackage{xcolor}
\usepackage{booktabs}

% Beamer theme settings
\usetheme{Madrid}
\usecolortheme{default}

\title{URL Coursework 2}
\subtitle{StackGAN++: Realistic Image Synthesis with Stacked Generative Adversarial Networks}
\author{Bruno Sánchez Gómez}
\date{\today}

\setbeamertemplate{footline}{%
    \leavevmode%
    \hbox{%
        \begin{beamercolorbox}[wd=0.2\paperwidth,ht=2.5ex,dp=1.125ex,center]{author in head/foot}%
            \hspace*{2mm}Bruno Sánchez Gómez
        \end{beamercolorbox}%
        \begin{beamercolorbox}[wd=0.7\paperwidth,ht=2.5ex,dp=1.125ex,center]{title in head/foot}%
            \insertshortsubtitle
        \end{beamercolorbox}%
        \begin{beamercolorbox}[wd=0.1\paperwidth,ht=2.5ex,dp=1.125ex,center]{date in head/foot}%
            \hspace*{-2mm}\insertframenumber{} / \inserttotalframenumber
        \end{beamercolorbox}%
    }%
    \vskip0pt%
}

\begin{document}

\begin{frame}
    \titlepage
\end{frame}


\begin{frame}{Table of Contents}
    \tableofcontents
\end{frame}

\section{Introduction}

\begin{frame}{Area of Research of the Paper}
    \centering
    \begin{minipage}{0.8\textwidth}
        \begin{block}{Definition}
            \centering
            More specific than Unsupervised Learning.
        \end{block}
    \end{minipage}
    \vspace{0.5cm}
    \begin{itemize}
        \item What is the problem that is addressed in the paper
        \item \textbf{Applications?}
            \begin{itemize}
                \item .
            \end{itemize}
        \item \textbf{Approaches?}
            \begin{itemize}
                \item .
            \end{itemize}
    \end{itemize}
\end{frame}

\begin{frame}{StackGAN++~\cite{stackgan++}}
    \begin{itemize}
        \item \textbf{StackGAN++:} What is the contribution of the paper. Basically, what is proposed on the paper that previous research does not address or what is the improvement
    \end{itemize}
\end{frame}

\section{StackGAN++ Methodology}
\begin{frame}{Table of Contents}
    \tableofcontents[currentsection]
\end{frame}

\begin{frame}{Components of StackGAN++}
    \begin{figure}
        \centering
        % \includegraphics[width=0.75\textwidth]{figures/inference_process.png}
    \end{figure}
    \begin{itemize}
        \item What are the minimum details that are necessary to understand how the proposed method works
    \end{itemize}
\end{frame}

\begin{frame}{Results}
    \begin{itemize}
        \item How the method has been tested (datasets, procedures) and what other competing methods have been used for comparison and why.
        \item Quantitative/Qualitative results?
    \end{itemize}
\end{frame}

\begin{frame}{Sblation Studies}
    \begin{itemize}
        \item If it is the case, what experiments have been done to test the different elements that compose the proposed methods (they are usually called ablation studies)
    \end{itemize}
\end{frame}

\section{Critical Analysis}
\begin{frame}{Table of Contents}
    \tableofcontents[currentsection]
\end{frame}

\begin{frame}{Pros}
    \begin{itemize}
        \item What is the improvement over the competing methods
    \end{itemize}
\end{frame}

\begin{frame}{Cons}
    \begin{itemize}
        \item What are the limitations of the proposed method
        \item Maybe Pros and Cons only 1 slide?
    \end{itemize}
\end{frame}

\begin{frame}{Conclusions}
    \begin{itemize}
        \item What conclusions can be taken after analysing the results
    \end{itemize}
\end{frame}

\begin{frame}{Q\&A}
    \centering
    \Huge{Thank you for your attention!}

    \vspace{0.5cm}

    \Large{Any questions?}
\end{frame}

\begin{frame}[allowframebreaks]{References}
    \bibliographystyle{unsrt} % Or choose another style like unsrt, alpha, etc.
    \bibliography{references}
\end{frame}

\end{document}